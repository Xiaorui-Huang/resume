%%%%%%%%%%%%%%%%%
% This is an sample CV template created using altacv.cls
% (v1.3, 10 May 2020) written by LianTze Lim (liantze@gmail.com). Now compiles with pdfLaTeX, XeLaTeX and LuaLaTeX.
%
%% It may be distributed and/or modified under the
%% conditions of the LaTeX Project Public License, either version 1.3
%% of this license or (at your option) any later version.
%% The latest version of this license is in
%%    http://www.latex-project.org/lppl.txt
%% and version 1.3 or later is part of all distributions of LaTeX
%% version 2003/12/01 or later.
%%%%%%%%%%%%%%%%

%% If you need to pass whatever options to xcolor
\PassOptionsToPackage{dvipsnames}{xcolor}

%% If you are using \orcid or academicons
%% icons, make sure you have the academicons
%% option here, and compile with XeLaTeX
%% or LuaLaTeX.
% \documentclass[10pt,a4paper,academicons]{altacv}

%% Use the "normalphoto" option if you want a normal photo instead of cropped to a circle
% \documentclass[10pt,a4paper,normalphoto]{altacv}

\documentclass[10pt,a4paper,ragged2e,withhyper]{altacv}

%% AltaCV uses the fontawesome5 and academicons fonts
%% and packages.
%% See http://texdoc.net/pkg/fontawesome5 and http://texdoc.net/pkg/academicons for full list of symbols. 

%You MUST compile with XeLaTeX or LuaLaTeX if you want to use academicons.

% Change the page layout if you need to
\geometry{left=1.25cm,right=1.25cm,top=1.5cm,bottom=1.5cm,columnsep=1.2cm}

% The paracol package lets you typeset columns of text in parallel
\usepackage{paracol}
\usepackage{graphicx,wrapfig}

% Change the font if you want to, depending on whether
% you're using pdflatex or xelatex/lualatex
\ifxetexorluatex{}
  % If using xelatex or lualatex:
  \setmainfont{Roboto Slab}
  \setsansfont{Lato}
  \renewcommand{\familydefault}{\sfdefault}
\else
  % If using pdflatex:
  \usepackage[rm]{roboto}
  \usepackage[defaultsans]{lato}
  % \usepackage{sourcesanspro}
  \renewcommand{\familydefault}{\sfdefault}
\fi

% Change the colours if you want to
\definecolor{Black}{HTML}{000000}
% \definecolor{SlateGrey}{HTML}{2E2E2E}
% \definecolor{LightGrey}{HTML}{666666}
% \definecolor{DarkPastelRed}{HTML}{450808}
% \definecolor{PastelRed}{HTML}{8F0D0D}
% \definecolor{GoldenEarth}{HTML}{E7D192}
\definecolor{DarkRed}{HTML}{8B0000}
\definecolor{LightRed}{HTML}{B00000}

\colorlet{name}{Black}
\colorlet{tagline}{Black}
\colorlet{heading}{DarkRed}
\colorlet{headingrule}{LightRed}
\colorlet{subheading}{Black}
\colorlet{accent}{Black}
\colorlet{emphasis}{Black}
\colorlet{body}{Black}

% Change some fonts, if necessary
\renewcommand{\namefont}{\Huge\rmfamily\bfseries}
\renewcommand{\personalinfofont}{\footnotesize}
\renewcommand{\cvsectionfont}{\LARGE\rmfamily\bfseries}
\renewcommand{\cvsubsectionfont}{\large\bfseries}


% Change the bullets for itemize and rating marker
% for \cvskill if you want to
\renewcommand{\itemmarker}{{\small\textbullet}}
\renewcommand{\ratingmarker}{\faCircle}

%% sample.bib contains your publications
\addbibresource{sample.bib}

\begin{document}

%% Depending on your tastes, you may want to make fonts of itemize environments slightly smaller
\AtBeginEnvironment{itemize}{\small}
%% Set the left/right column width ratio to 6:4.  \columnratio{0.6}

% Start a 2-column paracol. Both the left and right columns will automatically
% break across pages if things get too long.
\begin{paracol}{2}

    \name{Xiaorui Huang}
    \tagline{Always Fascinated \faLaptopCode}
    %% You can add multiple photos on the left or right
    % \photoR{2.8cm}{Globe_High}
    % \photoL{2.5cm}{Yacht_High,Suitcase_High}

    \personalinfo{%
        % Not all of these are required!
        % \printinfo{\faCalendar}{Availability: \hspace{0.5ex} \textbf{From May 2024}}
        \printinfo{\faUser}{Preferred Name: \textbf{Richard}}
        \email{richardxr.huang@mail.utoronto.ca}
        \phone{+1 (289) 772--8682}
        \hspace{1.7ex}\location{Toronto, Canada}
        %\homepage{www.pending.dev}

        \linkedin{xiaorui-richard-huang}
        \github{Xiaorui-Huang}
        %% You MUST add the academicons option to \documentclass, then compile with LuaLaTeX or XeLaTeX, if you want to use \orcid or other academicons commands.
        % \orcid{0000-0000-0000-0000}
        %% You can add your own arbtrary detail with
        % \printinfo{symbol}{detail}[optional hyperlink prefix]
        % \printinfo{\faPaw}{Hey ho!}[https://example.com/]
        %% Or you can declare your own field with
        %% \NewInfoFiled{fieldname}{symbol}[optional hyperlink prefix] and use it:
        % \NewInfoField{gitlab}{\faGitlab}[https://gitlab.com/]
        % \gitlab{your_id}
    }
    \makecvheader{}

    \vspace{-10pt}


    \cvsection{Experience}

    \cvevent{eAI Machine Learning Engineer}{Qualcomm}{May 2023 --- Aug 2023}{Markham, ON}
    \begin{itemize}
        \item Led efforts on \textbf{Neural Architecture Search (NAS)} and model compression within the \textbf{Edge AI R\&D} team.

              % \item Developed a NAS framework, leveraging Qualcomm's patented NAS
              %       techniques, to optimize \textbf{arbitrary models}\footnote*{NAS support
              %           is required for NN layers \hspace{0.6ex} \textit{E.g. nn.Conv2d is
              %               supported}} for \textbf{any profiled hardware}, harnessing Pytorch's
              %       \textbf{torch.fx} extensively.
        \item Developed a NAS framework, leveraging Qualcomm's patented NAS
              techniques, to optimize \textbf{arbitrary models} for \textbf{any profiled hardware}, harnessing Pytorch's
              \textbf{torch.fx} extensively.

        \item Streamlined the NAS workflow for incoming client models, slashing
              \textbf{engineering time} by \textbf{80\%}.

        \item Achieved a \textbf{50\% reduction} in \textbf{model size} and a
              \textbf{60\% drop in inference latency} without compromising accuracy across benchmark models.

        \item Engaged in lab meetings focused on cutting-edge model compression
              research, particularly \textbf{Quantization}.

        \item Delivered a comprehensive presentation on the NAS framework to the
              broader eAI team.

    \end{itemize}
    \cvtag{NAS}
    \cvtag{Quantization}
    \cvtag{Pytorch}
    \cvtag{torch.fx}
    \cvtag{ONNX}

    \divider{}

    \cvevent{RPA Backend Developer}{IBM}{May 2022 --- Apr 2023}{Markham, ON}

    \begin{itemize}

        \item Worked on backend development for IBM's Robotics Process Automation (RPA) platform, written in  \textbf{C\#}.

        \item Augmented IBM RPA's \textit{WAL} programming
              language, introducing a reflection feature resembling Java and C\#.

        \item Collaborated with cross-functional teams, achieving a \textbf{15\%} reduction in customer issues and defects per release.

        \item Employed \textbf{agile methodologies}, showed both independent and
              collaborative competencies in a hybrid environment.

        \item Articulated and presented solution strategies to RPA's senior architects and product teams.

              % \item Backend software development working on IBM Robotics Process Automation (RPA).

              % \item Increased IBM RPA's \textit{WAL} programming language usability by
              %     developing reflection feature similar to Java and \textbf{C\#}.

              % \item Reduced 10\% customer issues and product defects per release through
              %     collaboration with multi-disciplined teams. 

              % \item Conducted \textbf{agile} development process, demonstrated independent
              %     and teamwork skills in hybrid environment.

              % \item Presented solution to RPA senior architect \& product teams. 

    \end{itemize}
    \cvtag{C\#}
    \cvtag{Pragramming Language Design}
    \cvtag{Agile}

    \vspace{-3pt}
    \cvsection{Education}

    \cvevent{University of Toronto \faUniversity}{Candidate for HBSc.\ in Computer Science}{Sep 2019 --- Expected Jun 2024}{}
    % \vspace{-5pt}
    \begin{itemize}

        \item  CSC367 \textbf{Parallel Computing} (In Progress)
              Parallel Arch \& Algo, threading \& OpenMP, Distributed Computing w/ MPI,
              \textbf{CUDA Arch \& Reduction Algo}, Cloud Computing

              % \item CSC420 \textbf{Computer Vision (85\%)} --- Convolution, Feature
              %       Extraction (SIFT), Optical Flow, Feature Matching (RANSAC), Camera,
              %       Stereo, Homography, Object Detection

        \item CSC413 \textbf{Deep Learning} (96\%) --- \textbf{Transformers}, CNN, RNN, GAN, VAE, GNN, RL, Model Tuning techniques

        \item ECE568 \textbf{Computer Security} (83\%) --- Buffer Overflow \& Control Hijacking, \textbf{Cache Side-Channel} Attacks, Network Security, Cryptography, Web Security
              % \item  \textbf{CSC258 Computer Organization - 95\%} Implemented
              % and understood core computer component from basic circuits as well as
              % writing assembly instructions.

        \item CSC317 \textbf{Computer Graphics} (97\%) --- Ray Tracing, Mass
              Spring Systems, BVH, Meshes, Kinematics, OpenGL Shaders in
              \textbf{C++} using \textbf{Eigen} and \textbf{libigl}

              % \item \textbf{CSC209 Software Tools \& Sys Programming- 93\%} Bash Scripting, C, software tools, pipes, file processing, system calls
    \end{itemize}
    % \cvtag{NLP}
    \cvtag{CSC369 OS}
    \cvtag{CSC401 NLP}
    \cvtag{CSC412 Probabilistic ML}
    % \cvtag{Parallel Computing}
    % \cvtag{Computer Security}
    % \cvtag{C++}
    % \cvtag{Pytorch}
    % \cvtag{Linear Algebra}
    % \cvtag{Algorithms}
    % \cvtag{Stats \& Probablity}
    % \cvtag{Multivariate Calc}
    % \vspace{-5pt}
    % \cvevent{}{Other Courses}{}{}
    % Linear Algebra, Statistic and Probability, Multivariate Calculus, Algorithm Design and Complexity

    \switchcolumn{}

    \cvsection{Research}
    \cvevent{Machine Learning Reseach Intern}{embARC Research Group}{Jan 2024 --- Now}{University of Toronto}
    \begin{itemize}
        \item Research on \textbf{real-time Gaussian Splatting} \& \textbf{NeRF} 3D reconstruction with data captured on embedde devices.
        \item Provides incremental Point Cloud initialization and dataset sampling techniques to improve real-time reconstruction performance.
        \item Supervised by \href{https://www.cs.toronto.edu/~nandita/}{\textbf{Prof. Nandita Vijaykumar}}
    \end{itemize}

    \cvtag{3D Gaussian Splatting} \cvtag{SLAM} \cvtag{NeRF} \cvtag{Pytorch} %\cvtag{CUDA}

    \divider{}

    \cvevent{Linearly Explored Learning Rate Scheduler}{}{Apr 2022 \github{RolandGao/pycls}}{}
    \begin{itemize}

        \item We introduced the LES method to automate and refine the resource-intensive task of \textbf{learning rate tuning}.

        \item LES achieves a final error rate of 8\% on par with other commonly used
              optimizer and schedulers on \href{https://github.com/facebookresearch/pycls}{pycls} code base
              \textbf{without the need for learning rate tuning}.

        \item Developed a custom \textbf{SGD with momentum} algorithm to facilitate exploration of various backpropagation strategies during LES creation.


    \end{itemize}
    % \cvevent{IBM RPA Backend Intern}{The 31st World Hakka Conference}{May 2021 -- Aug 2021}{Toronto, ON}
    % \begin{itemize}
    %     \item Full-Stack software development building an Event Management app; streamlining registration process and attendee data retrieval 

    %     \item Designed the app with a cross-disciplinary team to address limitations with WordPress, built solutions and features using Django and Vue.js 

    %     \item Conducted usability testing, moderated test sessions with users and design teams. Brainstormed and applied new ideas using Design Thinking

    %     \item Presented and demonstrated solutions to senior executives
    % \end{itemize}

    %\divider

    %\cvevent{Waiter, Service Crew}{Xiao Liang Zheng Xia}{June 2019 -- August 2019}{Wuhan, China}

    %\begin{itemize}
    %\item Full-stack software development on a talent management internal app                   
    %\item Built proofs-of-concept and pitched new software features

    %\item Completed tasks in a cross-disciplinary team through the Agile workflow

    %\item Presented and demonstrated solutions to senior executives
    %\end{itemize}

    \cvsection{Projects}

    % \begin{wrapfigure}[1]{r}[-25pt]{1.5cm}
    % \includegraphics[width=1.5cm]{icon_transparent.png}
    % \end{wrapfigure} 


    %% Switch to the right column. This will now automatically move to the second
    %% page if the content is too long.

    \cvevent{\textbf{CUDA} Ray Tracing}{Almost Real Time Ray Tracing}
    {Nov 2023 \github{Xiaorui-Huang/cuda-ray-tracing}
    }{}
    \begin{itemize}
        \item Implemented a \textbf{CUDA} ray tracer with \textbf{BVH} acceleration structure, with \textbf{Blinn-Phong} shading.
        \item Achieved \textbf{real-time} ray-tracing of \textbf{30 FPS} and \textbf{2000x Speedup} on RTX3060-Ti compared to CPU\@.
        \item Incorporated dynamically loaded Scene generation to allow for future interactivity.
    \end{itemize}
    \cvtag{CUDA}
    \cvtag{C\//C++}
    \cvtag{Computer Graphics}

    % \divider{}
    % \vspace{-2pt}

    % \cvevent{Eedimator}{Online course performance predictor}{Nov 2021
    % \github{Xiaorui-Huang/Eedimator}
    % }{}
    % \begin{itemize}
    %     \item A \textbf{predictor of students' ability to answer questions}, based on previous answers and other students' answers allowing online education platforms to provide tailored assistance.
    %     \item Used Machine Learning algorithms such as \textbf{Neural Networks, Matrix Factorization, Item Response Theory, and K-NN} to create an \textbf{ensembled} prediction model.
    %     \item Based on \textbf{NeurIPS} 2020 Education Challenge and uses real-world data collected on  \printinfo{\faGlobe} {eedi.com} [https://eedi.com/projects/neurips-education-challenge] 

    % \end{itemize}
    % \cvtag{Python}
    % \cvtag{Pytorch}
    % \cvtag{numpy}
    % \cvtag{scikit-learn}

    \divider{}


    \cvevent{Woodoku Learn}{Reinforcement Learning Model }{Jul 2022
        \github{EdwardHaoranLee/WoodokuLearn}
    }{}
    \begin{itemize}

        \item Replicated the mobile game \href{https://play.google.com/store/apps/details?id=com.tripledot.woodoku&hl=en_CA&gl=US}{Woodoku} for the terminal using Python, enabling both human and AI gameplay through dedicated environment APIs.
        \item Employed Q-Learning, a \textbf{Reinforcement Learning} approach with Pytorch, targeting top scores on the Woodoku leaderboard.
              % \item Adhered to \textbf{agile} methodologies; integrated CI testing, static type checks, and employed tools like GitHub Actions, pytest, and mypy for efficient code reviews and development.

    \end{itemize}
    % \cvtag{python}
    \cvtag{RL}
    \cvtag{Pytorch}
    \cvtag{OOP}
    % \cvtag{mypy}
    \cvtag{Agile}
    \cvtag{CMake}

    % \begin{wrapfigure}[1]{r}[-25pt]{1.5cm}

    % \vspace{-15pt}

    % \includegraphics[width=1.5cm]{guy_smiling.png}
    % \end{wrapfigure} 

    % \divider
    %\newpage

    % \cvevent{Simple Gift Shop}{Checkout Calculator}
    % {Sep 2021 
    % \github{Xiaorui-Huang/Simple-Gift-Shop}
    % }{} 
    %   %% \printinfo{symbol}{detail}[optional hyperlink prefix]
    % \begin{itemize}
    %     \item A simple check out calculator for an array of mock gadgets 
    %     \item Front-end build with React and styled with tailwindcss, Backend built with Express.js connecting to MongoDB
    %     \item Practiced Agile methodology and DevOps workflow using GitHub Action and Deployed with Heroku
    % \end{itemize}

    % \cvtag{React}
    % \cvtag{Express.js}
    % \cvtag{CI/CD}
    % \cvtag{MongoDB}
    % \cvtag{tailwindcss}
    % \cvtag{Heroku}
    % \cvtag{GitHub Action}
    % \cvtag{Agile}

    % \cvevent{North Arrow}{University Enrollment Advisor}
    % {June 2018 
    % \github{Xiaorui-Huang/North-Arrow}
    % }{} 
    %   %% \printinfo{symbol}{detail}[optional hyperlink prefix]
    % \begin{itemize}
    %     \item A University Enrollment Advisor for South African Universities, provides program recommendations using user's academic records
    %     \item Developed Front-End with Swing and integrated with SQL 

    % \end{itemize}
    % \cvtag{Java}
    % \cvtag{Java Swing}
    % \cvtag{JFreeChart}
    % \cvtag{SQL}

    %\medskip

    % \cvsection{A Day of My Life}

    % % Adapted from @Jake's answer from http://tex.stackexchange.com/a/82729/226
    % % \wheelchart{outer radius}{inner radius}{
    % % comma-separated list of value/text width/color/detail}

    % \wheelchart{1.5cm}{0.5cm}{%
    %   6/8em/accent!30/{Sleep,\\beautiful sleep},
    %   3/8em/accent!40/Hopeful novelist by night,
    %   8/8em/accent!60/Daytime job,
    %   2/10em/accent/Sports and relaxation,
    %   5/6em/accent!20/Spending time with family
    % }

    % use ONLY \newpage if you want to force a page break for
    % ONLY the current column
    %\newpage

    % \cvsection{Publications}

    % \nocite{*}

    % \printbibliography[heading=pubtype,title={\printinfo{\faBook}{Books}},type=book]

    % \divider

    % \printbibliography[heading=pubtype,title={\printinfo{\faFile*[regular]}{Journal Articles}},type=article]

    % \divider

    % \printbibliography[heading=pubtype,title={\printinfo{\faUsers}{Conference Proceedings}},type=inproceedings]

    %% Switch to the right column. This will now automatically move to the second
    %% page if the content is too long.
    % \switchcolumn

    % \cvsection{My Life Philosophy}

    % \begin{quote}
    % ``Something smart or heartfelt, preferably in one sentence.''
    % \end{quote}

    % \cvsection{Awards}

    % \cvachievement{\faTrophy}
    % {New Hacks 2020, 2nd Place Overall}
    % {Hackathon hosted by IEEE UofT. SBC, Hardware peripherals, Speech to Text API, with Project Boomba
    % %Top 3\% percentile globally; Online Cyber-Security Contest hosted by Carnegie Mellon University. Forensics, web exploitation, cryptography, reverse engineering
    % \newline
    %  \printinfo{\faGlobe}
    % {Boomba on devpost.com}
    % [https://devpost.com/software/bomba-fucking-shut-up-simulator]
    % }
    % \divider

    % \cvachievement{\faMedal}{SA Mathematics Olympiad, 3rd Round}
    % {Participated the 3rd round of South African Mathematics Olympiad}

    % \divider

    % \cvachievement{\faAward}{Top 5\% in Math, Physics and Information Technology}
    % {Top 5\% in South African Secondary schools for STEM subjects
    % }

    %\divider

    %\cvachievement{\faChartBar}{DMOJ::Modern Online Judge}{Various online competitive programming contests and problems regarding algorithm design. I also set problems and contribute to DMOJ open source
    %\homepage{dmoj.ca/user/ManchurioX/solved}
    %\github{kevinjycui/Competitive-Programming}}

    \cvsection{Skills}
    \vspace{-5pt}
    \cvevent{}{Programming Languages}{}{}
    \cvtag{\faPython{}Python}
    \cvtag{\textbf{C/C}++}
    \cvtag{CUDA}
    \cvtag{C\#}
    \cvtag{\faJava{} Java}
    \cvtag{\faRust{} Rust}\\
    \cvtag{\LaTeX}
    % \cvtag{PowerShell}
    \cvtag{R}
    \cvtag{TypeScript}
    \cvtag{HTML\&CSS}
    \cvtag{SQL}
    % \cvtag{Bash Scripts}


    %\medskip
    % \divider

    % \cvevent{}{Development Libraries\& Environment}{}{}
    % \cvtag{Google Cloud API}
    %\cvtag{Node.js}
    %\cvtag{Maven}
    %\cvtag{GNU}

    %\medskip
    \divider{}

    \vspace{-5pt}
    \cvevent{}{Skills, Frameworks \& Development Environments}{}{}
    \cvtag{3D Reconstruction}
    \cvtag{Model Compression}
    \cvtag{Parallel Algorithms}
    \cvtag{Pytorch}
    % \cvtag{torch.fx}
    \cvtag{Vim}
    \cvtag{\faDocker{}Docker}
    % \cvtag{ROS}
    % \cvtag{TensorFlow}
    % \cvtag{React}
    % \cvtag{Vue.js}
    % \cvtag{Django}
    % \cvtag{MongoDB}
    % \cvtag{Express.js}
    % \cvtag{tailwindcss}
    \cvtag{\faLinux{}WSL}
    \cvtag{\faGit{}}
    \cvtag{VSCode}
    % \cvtag{Visual Studio}
    % \cvtag{AWS}
    % \cvtag{Azure}

    %\medskip
    % \divider


    % \cvevent{}{Concepts \& Practices}{}{}
    % \cvtag{OOP}
    % \cvtag{Agile}
    % \cvtag{REST API}
    % \cvtag{DevOps}
    %\cvtag{Design Thinking}

    % \divider
    % \cvevent{}{Hobbies}{}{}
    % \cvtag{Running}
    % \cvtag{Cycling}
    % \cvtag{Swimming}
    % \cvtag{Rock Climbing}
    % \cvtag{Coffee\faCoffee}
    % \cvtag{Music - Lots of Music\faMusic\href{https://open.spotify.com/user/xbhxp7y4k45jdxk9oh08vo1ot?si=0a3ab7340f8f44e4}{\faSpotify}}

    \bigskip

    % \textit{Idiomatic in English and in Mandarin Chinese}

    % \cvsection{Languages}

    % \cvskill{English}{5}
    % \divider

    % \cvskill{Spanish}{4}
    % \divider

    % \cvskill{German}{3}

    %  Yeah I didn't spend too much time making all the
    %  spacing consistent... sorry. Use \smallskip, \medskip,
    %  \bigskip, \vpsace etc to make adjustments.
    % \medskip


    % \cvsection{Education}

    % \cvevent{University of Toronto}{Candidate for B.Sc.\ in Computer Science}{2019 -- Ongoing}{Toronto, ON}
    % \cvevent{}{Relevant Courses}{}{}
    % \begin{itemize}
    %     \item  \textbf{CSC311 Machine Learning} Pytorch, Linear and Logistic regression, Deep Neural Net, CNN, Principle Component Analysis, Reinforcement Learning. Implemented concepts in Python from scratch or with help from libraries.
    %     \item  \textbf{CSC258 Computer Organization - 95\%} Implemented and understood core computer component from basic circuits as well as writing assembly instructions.

    %     \item \textbf{CSC209 Software Tools \& Sys Programming- 93\%} Bash Scripting, C, software tools, pipes, file processing, system calls
    % \end{itemize}

    % \textit{Other Courses}: Algorithms and Data structures, Statistic and Probability

    % \divider

    % \cvsection{Referees}

    % % \cvref{name}{email}{mailing address}
    % \cvref{Prof.\ Alpha Beta}{Institute}{a.beta@university.edu}
    % {Address Line 1\\Address line 2}

    % \divider

    % \cvref{Prof.\ Gamma Delta}{Institute}{g.delta@university.edu}
    % {Address Line 1\\Address line 2}


\end{paracol}


\end{document}

